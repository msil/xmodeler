\documentclass{article}

\title{XMF Features}

\author{Ceteva Ltd}

\begin{document}

\maketitle

XMF is a generic executable metamodelling facility for defining and deploying
semantically rich (executable) languages. These languages can be tailored to
specific domains, or can be widely applicable general purpose languages, for 
example mapping languages. XMF builds on and extends MDA standards such as 
MOF, UML, OCL and QVT, thus providing an intuitive and powerful foundation for
realising model-driven development.

XMF has been populated with a powerful collection of technologies for
language definition (metamodelling) and modelling in general. Some of the
core technologies are described below.

\begin{description}

\item[Aspects]
XMF allows existing models to be extended with {\em aspects}. An aspect is
a collection of definitions that are added to an existing model. The
aspect is modelled in its own right and contains the new definitions. For
example, adding a {\em toXML} aspect to a collection of classes.

\item[Assembler]
XMF contains its own machine code definition and assembler. Retaining control
in this way allows the machine code to be extended with new instructions and
to introduce debugging information in the running object code. Compilers for
new languages can compile directly to the XMF assembler. In this way XMF provides
a flexible language development system.

\item[Associations]
Binary associations are a convenient abstraction over attributes. Defining
associations involves adding a pair of attributes to classes and an invariant
between them.

\item[Clients]
XMF can connect to external processes using its {\em clients} interface. XMF
can act either as a server or a client and can connect to multiple external
processes. The external processes can be managed by creating appropriate 
internal XMF processes.

\item[Constructors]
XMF provides flexible mechanisms for creating instances of classes. A class
can include the definition of BOA-constructors that define how a class can be
used as an operator to create an instance. Keyword constructors can be used to
create an instance of a class based on explicit slot initialization.

\item[Compiler]
XMF implements its own compiler. The compiler is bootstrapped. The compiler
can easily be extended or even replaced to support a range of languages.

\item[Constraints]
XMF provides constraints that are attached to classifiers. A constraint is
an invariant that applies to all instances of the classifier. Execution of
constraints against a candidate instance is performed under program control
and produces a {\em constraint report} that describes whether the constraints
were successful. In the case that a constraint fails, the report provides
information that describes why.

\item[Daemons]
XMF supports {\em daemons} that are attached to objects in order to monitor 
slot values. Daemons can be programmed to perform actions when slot values
change. XMF provides a number of abstractions based on daemon technology.
For example, deamons are the basis of synchronizing multiple objects in a
data driven environment.

\item[Data Types]
XMF provides the following basic data types:
\begin{description}
\item[Boolean] True or false.
\item[Enumeration] Enumerated values.
\item[Float] Floating point numbers.
\item[Integer] Integer numbers.
\item[Null] The undefined value.
\item[Object] A value with a classifier and named slots. Most things are objects in XMF.
\item[Operation] A value that can be applied to arguments in order to perform an action and
return a result. Operations are first class values in XMF and observe lexical scoping.
\item[Sequence] Either the empty sequence or a sequence eith a head (arbitrary value) and a tail 
(a sequence). A sequence has identity so that the head and tail of a sequence may be updated.
\item[Set] A set of values; a set has no identity.
\item[String] A sequence of characters; a string has no identity.
\item[Symbol] A string with identity. Two symbols with the same sequence of characters has the same identity. 
Typcially, symbols are used for XMF names to make lookup efficient.
\item[Table] A hash table.
\item[Vector] A fixed length sequence of values.
\end{description}
All other XMF data values are implemented in terms of these basic types. Object is particularly
heavily used and is the basis of XMF meta-circularity.

\item[Delivery Edition]
XMF supports a delivery mode whereby the engine runs executable models
developed by a user (possibly using the Developer's Edition) and saved
using one of the XMF save formats. Typically the delivery edition will be
much smaller than the developer's edition and contain application specific
executable models. This mode is suitable for deploying XMF as an embedded 
system.

\item[Developer's Edition]
XMF supports a development mode whereby the engine runs Ceteva models
for constructing and developing languages, tools and models. The developer's
edition contains tools for constructing diagrams and for inspecting and
editing model elements. 

\item[Dialogs]
The XMF Dialog language provides a declarative way to construct interactive
text based dialogs with a user. The language abstracts away from the mechanisms
by which the input and output occurs, allowing the developer to focus on the
information that must be conveyed by the dialog.

\item[Documentation]
XMF allows elements to be declared as being self documenting. Documentation
can be then entered when elements (such as packages, classes and operations)
are defined. The basic XMF system is self documenting. Aspects are defined
to translate the documentation to a variety of formats including HTML.

\item[Engine]
The XMF engine is a virtual machine that executes XMF machine code produced 
by the XMF compiler and assembler. The engine can be initialised with, and
can produce, saved images. A saved image is saved on backing storage and can
be reloaded into any XMF engine.

\item[Execution]
XMF modelling languages are {\em executable}. Instances of classes can
be created and their state can be updated. Execution is achieved by compiling
an executable model to XMF assembler or by using the XMF interpreter. Since
XMF is a {\em meta-programming environment} it is easy to construct new executable
modelling languages. Execution in a new language is achieved by: translating
from the source of the new language to the source of an existing XMF language;
or by compiling the new language directly to XMF assembler; or by extending
existing XMF executable language constructs; or by writing a new interpreter
for the language. The choice of these strategies will depend on how close the new
language is to an existign XMF language and whether the new language needs to
execute in a highly efficient embedded mode (compiled code being more efficient
than interpreted code).

\item[Exceptions]
XMF provides a Java-like exception handling mechanism. Exception can be thrown
in the XMF machine, in system code or in user code. Handlers can be defined
to catch any of these exceptions and to take appropriate action.

\item[Flexible Input and Output]
XMF provides {\em channels} that are used for input and output. Channels can be
attched to strings, files or external processes. XMF provides a programmable
{\em format} that is used with any output channel; this uses format characters
to provide very flexible control over output. Token input channels are provided
that recognize sequences of input characters and transform them to data
structures such as integers and strings.

\item[Garbage Collection]
The XMF engine manages its own heap and performs its own garbage collection.
XMF applications do not need to manage memory and the XMF garbage collector
can be tailored to suit the specific needs of different applications.

\item[Java]
XMF is written in Java and is therefore highly portable. The basic XMF engine makes 
very little use of the Java libraries which, depending on the number of XMF
packages loaded, provides XMF with a very small footprint.

\item[Language Definition]
XMF provides packages for language definition. These include the XBNF grammar
definition language and Quasi-Quotes that support pattern-based source language
mappings. XMF is provided with an extensive executable modelling language that 
has been defined using these features: XOCL. XOCL (and any other language 
engineered using XMF) can be seamlessly extended by defining new syntax
classes. A syntax class defines a grammar for parsing the concrete syntax of the
new construct and provides mechanisms for executing the new construct. Library
classes (Sugar etc) make this process very easy for most cases.

\item[Mappings]
XMF provides mapping technologies. A mapping is a structured definition that is applied
to a value to produce a new value. XMF mappings use pattern matching to determine
whether and how they should match and transform supplied values. Since XMF is meta-circular
mappings may be supplied with values, models or meta-models. XMF mappings are themselves
values and are therefore identifiable components that can be constructed, manipulated 
and saved in their own right. 

\item[Meta-Circularity]
XMF is {\em meta-circular}. This means that XMF contains a complete definition 
of itself (or equivalently XMF is written in XMF). There are a variety of benefits
from this: XMF is its own test suite; XMF is intended to be a language engineering
environment -- this claim is validated by its meta-circularity; since XMF is
based on an OO model, it is extensible. Everything in XMF is either a simple
value, a collection or an object. This uniformity of representation makes writing
meta-tools in XMF very easy. For example, XMF provides packages for translating 
data values to XML; since everything is ultimately one of a handful of data value types,
this package can produce XML from any XMF value (even the XML package itself).

\item[Modelling]
XMF provides a range of basic modelling abstractions. All of the abstractions are
defined using the language definition technologies provided by XMF. The abstractions
include:
\begin{description}
\item[Association]
A pair of attributes and a constraint between them.
\item[Attribute] 
Defined in a class: defines the name and type of slots in the instances of the class.
\item[Class]
Defines the structure (attributes), behaviour (operations) and constraints
of objects that are created as instances of the
class. Classes have parents from which they inherit structure, behaviour and constraints.
\item[Constraint]
A constraint is a predicate that applies to instances of a given class.
\item[NameSpace]
A name space is a container of named elements. A package is a name space for packages and
classes. A class is a name space for attributes, operations and constraints. A name space
can import other name spaces in order to access their contents.
\item[Operation]
An operation has parameters and an action. An operation is invoked by supplying it with
parameter values. Typically an operation is invoked by sending a message to an object.
\item[Package]
A package is a container for sub-packages and classes.
\end{description}

\item[OCL and XOCL]
XMF provides a version of the OCL language and extends it with new features to
provide XOCL. OCL is the Object Constraint Language of the UML. Within UML it is used
to define relational features such as constraints and pre/post-conditions. OCL
provides a number of useful abstractions for language engineering such as collections
and iteration expressions. OCL is not, however, a programming language. XMF extends
OCL with imperative features and integrates it within the XMF meta-circular kernel
to provide a powerful programming language (XOCL) that can be used for modelling or for
language engineering. XOCL is defined in pure XMF and is therefore an example of 
a language that can be engineered using XMF technologies. Developers can take all
or part of XOCL and use it as the basis of new languages that are tailored to 
specific application domains. The basic extensions to OCL in XOCL include the following:
\begin{description}
\item[Case] Case statements are used to dispatch on a value. The clauses of the case statement
use pattern matching.
\item[Exception Handling] Try/catch-clauses and throw statements are used to manage exceptions.
\item[Find] Collections are searched using find statements. 
\item[For] Collections are traversed using for statements.
\item[Operations] Operations are first class data values in XOCL.
\item[Pattern Matching] Operation arguments and various language features employ pattern matching.
\item[Side Effects] Local variables, object slots, vectors, tables and sequences may be
changed by side effect.
\item[TypeCase] The type of a value can be interrogated using a type case statement.
\end{description}

\item[Operations]
XMF provides operations as a first class data structure. Together with other language
engineering technologies, first class operations are particularly effective when
defining new language abstractions.

\item[Pattern Matching]
XMF supports pattern matching. Patterns can occur in a variety of
language constructs such as {\em case} statements and operation parameters.
Pattern matching allows the developer to focus on information rather than
how to navigate to the required information.

\item[Processes]
The XMF VM XMF supports {\em multi-threading}. Threads do not block on input and 
output and are therefore useful for controlling connections to multiple
external processes.

\item[Save Formats]
XMF can save  and exchange models in XML, code and binary formats.

\item[Sugar]
XMF provides technologies that support domain specific languages through the use
of {\em syntactic sugar}. A new language feature is syntactic sugar when its
definition can be completely explained in terms of a translation to existing 
constructs. Sugar is very important when engineering language features that
abstract away from the mechanics of {\em how} the feature is achieved and provides
a focus on {\em what} the feature is providing for the user. Typical examples
of sugar provide bespoke language constructs that meet the users need for
expressing orthogonal aspects of a definition.

\item[Synchronisation]
XMF provides a synchronisation language for capturing rules that determine
when two or more element values have changed. When the conditions on these
rules are true, arbitrary actions can be called. These can be used to 
synchronise values across multiple model elements, and are generally useful
whenever the state of a particular model element is to be montiored.

\item[Top Level]
XMF (developer's edition) includes a powerful command interpreter that is used
to drive the XMF engine. The top level command interpreter can be used with or
without the XMF diagram clients provided by XMF-Mosaic.

\item[Walkers]
A {\em walker} is an object that is programmed to pass over a data structure and 
perform a given task. Walking technology relies on a predefined data format. Unlike
other sophistcated modelling environments, XMF provides a simple uniform meta-circular
data format. This form makes walking technology very easy to construct and very 
powerful. For example, a walker can be defined that passes over the entire XMF
object repository and searches for a value matching a given condition (expressed 
as an XMF operation).

\item[XBNF]
XBNF is a grammar language provided by XMF. XBNF is like EBNF except that an XBNF
grammar is attached to a XMF classifier in order to provide a new language construct 
that is seamlessly integrated with the XOCL language. The actions of an XBNF grammar
are used to synthesize XMF values during the parse. The actions are written in XOCL.
All language features in XMF are written in XBNF.

\item[XML]
XMF provides a sophisticated collection of XML technologies. Declarative rules can 
be attached tomodel elements that express how the elements are serialized to XML.
The serialization technology takes care of sharing by including XML references at
specific points in the XML tree. XMF provides XML parsing technology (like XBNF)
that allows XML input channels to be parsed and synthesized. Together these technologies
make it simple to integrate XMF with other software devlopment tools.

\end{description}

\end{document}