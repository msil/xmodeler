\documentclass{article}
\usepackage{fancyhdr}

\pagestyle{fancy}
\lhead{}
\chead{A Primer for XOCL}
\rhead{\thepage}
\cfoot{(c) 2004 Ceteva Ltd.}

\title{A Primer for XOCL\\(Incomplete Draft)}

\author{Ceteva Ltd.}

\begin{document}

\maketitle

\section{Introduction}

XMF is an environment for domain specific language design. XMF can be used
to define and deploy both diagrammatic languges such as UML and text-based
languages. The kernel of XMF is defined in a language called XOCL. XOCL
is the language that is used to define XMF and as such is the basic language
that XMF developers will use to develop XMF applications including new
domain specific languages. 

This document is a primer for XOCL. The approach is informal and we aim to 
introduce features of the language by example, occasionally using features 
in examples before they are completely introduced.

\section{Basic Data Types and Expressions}

XOCL provides basic data types for integers and booleans. Operations for
arithmetic ($+$, $-$, $/$ and $*$) are supported in addition to the usual
relational operators ($<$, $>$, $=$, $<=$ and $>=$). Boolean logic is supported
by infix operators {\tt and} and {\tt or} and prefix {\tt not}. The following example
shows an operator definition for factorial. The operator is named {\tt fact},
takes a single argument {\tt n} and is defined in the global context {\tt Root}
which means that the name {\tt fact} is available everywhere:
\begin{verbatim}
context Root
  @Operation fact(n)
    if n = 0
    then 1
    else n * fact(n - 1)
    end
  end
\end{verbatim}
Another example of a global operation definition is {\tt gcd} below that computes
the greatest common divisor for a pair of positive integers. The example shows that
operations can optionally have argument and return types:
\begin{verbatim}
context Root
  @Operation gcd(m:Integer,n:Integer):Integer
    if m = n 
    then n
    else
      if m > n 
      then gcd(m-n,n)
      else gcd(m,n-m)
      end
    end
  end
\end{verbatim}
Integers are represented in 24 bits in XMF. The operators {\tt and} and {\tt or} are
overloaded to perform bit comparison when supplied with integers. The operators {\tt lsh}
and {\tt rsh} are defined on integers. They take the number of bits to shift left and
right respectively. The following operation adds up the number of bits in an integer:
\begin{verbatim}
context Root
  @Operation addBits(n:Integer):Integer
    if n = 0 or n < 0
    then 0
    else 
      (n and 1) + (addBits(n.rsh(1)))
    end
  end
\end{verbatim}
In addition, integers support the follosing operators: {\tt mod}, {\tt abs}, {\tt bit},
{\tt max}, {\tt min} and {\tt byte}.

XMF supports floating point numbers. All numeric operators described above are defined
for floating point numbers. In general integers and floats can be mixed in expressions
and the integer is translated to the equivalent float. You can construct a float using the
constructor {\tt Float} so that {\tt 3.14} is created by {\tt Float(3,14)}. Floats
are translated to integers by the operations {\tt round} (upwards) and {\tt floor} (downwards).

Integer division is performed by the operation {\tt div} and floating point division is
performed by the infix operator {\tt /} (translating integers to floats as necessary).
The operators {\tt sqrt}, {\tt sin} and {\tt cos} are defined for floats. 

All values in XMF support the operation {\tt of} that returns the most specific class of 
the receiver. A value can be asked whether it is of a given class using the operator {\tt isKindOf}.
Classes can be compared using {\tt inheritsFrom} where a class is considered to inherit 
from itself. We could define {\tt isKindOf} as:
\begin{verbatim}
context Element
  @Operation isKindOf(type:Classifier):Boolean
    self.of().inheritsFrom(type)
  end
\end{verbatim}
The distinguished value {\tt null}, of type {\tt Null}, is special in that it returns
{\tt true} when asked whether it {\tt isKindOf} any class. It is used as the default value
for all non-basic classes in XMF.

The following operation returns {\tt true} when supplied with a number:
\begin{verbatim}
context Root
  @Operation isNum(x)
    (x.isKindOf(Integer) or
     x.isKindOf(Float)) and
    x <> null
  end
\end{verbatim}

\section{Strings, Symbols and Equality}

XMF strings are of type {\tt String}. The following operation wraps the string {\tt "The"} and
{\tt "."} around a supplied string:
\begin{verbatim}
context Root
  @Operation makeSentence(noun:String):String
    "The " + noun + "."
  end
\end{verbatim}
Strings are sequences of characters indexed starting from 0. Equality of strings is
defined by {\tt =} on a character by character comparison. Characters are represented
as integer ASCII codes. The following operation checks whether a string starts with an
upper case char:
\begin{verbatim}
context Root
  @Operation startsUpperCase(s:String):Boolean
    if s->size > 0
    then
      let c = s->at(0)
      in "A"->at(0) <= c and c <= "Z"->at(0)
      end
    else false
    end
  end
\end{verbatim}
Strings can be compared using {\tt <}, {\tt <=}, {\tt >} and {\tt >=} in which case the usual
lexicographic ordering applies.

Since strings are compared on a character by character basis this makes string comparison
relatively inefficient when performing many comparisons. Strings are often used as the
keys in lookup tables (for example as the names of named elements). In order to speed up
comparison, XMF provides a specialization of {\tt String} called {\tt Symbol}. A symbol is
the same as a string except that two symbols with the same sequence of characters have the
same identity. Comparison of symbols by identity rather than character by character is much
more efficient. A string {\tt s} can be converted into a symbol by {\tt Symbol(s)}.

Any value can be converted into a string using the operation {\tt toString}. To get the
string representation of a number for example: {\tt 100.toString())}.

\section{Tables}

A table is used to associate keys with values. A table has operations to add associations
between keys and values and lookup a given key. A table is created using the {\tt Table}
constructor supplying a single argument that indicates the approximate number of elements
to be stored in the table. Suppose that a library maintains records on borrowers:
\begin{verbatim}
context Root
  @Class Library
    @Attribute borrowers : Table = Table(100) end
  end
\end{verbatim}
A new borrower is added by supplying the id, name and address. When the new borrower is added
we check that no borrower has been allocated the same id. If the id is not already in use then
we register the new borrower by associating the id with a borrower record in the table:
\begin{verbatim}
context Library
  @Operation newBorrower(id:String,name:String,address:String)
    if not borrowers.hasKey(id)
    then
      let borrower = Borrower(id,name,address)
      in borrowers.put(id,borrower)
      end
    else self.error("Borrower with id = " + id + " already exists.")
    end
  end
\end{verbatim}
The library also provides an operation that gets a borrower record:
\begin{verbatim}
context Library
  @Operation getBorrower(id:String):Borrower
    if borrowers.hasKey(id)
    then borrowers.get(id)
    else self.error("No borrower with id = " + id)
    end
  end
\end{verbatim}
Tables provide operations to get all the keys and all the values:
\begin{verbatim}
context Library
  @Operation idsInUse():Set(String)
    borrowers.keys()
  end
  
context Library
  @Operation allBorrowers():Set(Borrower)
    borrowers.values()
  end
\end{verbatim}

\section{Sets}

A set is an unordered collection of elements. The elements of a set need not all be of the same type.
When {\tt T} is the type of each element of the set then the set is of type {\tt Set(T)}. Operations
are provided to add and remove elements from sets and to combine sets. Sets can be used in iterate
expressions. This section describes set-specific operations; iteration is described in section 
\ref{iteration}.

Sets are created by evaluating a set expression of the form: {\tt Set\{x,y,z,...\}} where {\tt x},
{\tt y}, {\tt z} etc are element expressions. For example:
\begin{verbatim}
Set{1,true,Set{"a","b","c"},Borrower("1","Fred","3 The Cuttings")}
\end{verbatim}
The expression {\tt Set\{\}} produces the empty set.
A set is unordered. An element can be selected at random from a non-empty set by performing {\tt S->sel}.
A set is empty when {\tt S->isEmpty} produces true (or when it is {\tt =} to {\tt Set\{\}}).
An element {\tt e} is added to a set by {\tt S->including(e)} and removed from a set by 
{\tt S->excluding(e)}. The union of two sets is produced by {\tt S1 + S2} and the difference
is constructed by {\tt S1 - S2}. An element is contained in a set when {\tt S->includes(e)}.

Suppose that the set operation {\tt includes} was not provided as part of XOCL. It could be defined
by:
\begin{verbatim}
context Set(Element)
  @Operation includes(e:Element):Boolean
    if self->isEmpty
    then false
    else 
      let x = self->sel
      in if x = e
         then true
         else self->excluding(x)->includes(e)
         end
      end
    end
  end
\end{verbatim}

\section{Sequences}

A sequence is an ordered collection of elements. The elements in the sequence need not all be of the
same type. When {\tt T} is the type of each element in the sequence then the sequence is of type
{\tt Seq(T)}. Sequences can be used in iterate expressions as described in section \ref{iterate}.
This section describes sequence operations.

Sequences are created by evaluating a sequence expression or by translating an existing element into 
a sequence. Sets, strings, integers and vectors can be translated to sequences of elements, characters,
bits and elements respectively by performing {\tt e.asSeq()}. 

The following operations are defined on sequences: {\tt +} appends sequences; {\tt asSet} transforms
a sequence into a set; {\tt asString} transforms a sequence of character codes into a string; {\tt asVector}
transforms a sequence into a vector; {\tt at} takes an index and returns the element at that position
in the sequence, it could be defined as:
\begin{verbatim}
context Seq(Element)
  @Operation at(n:Integer):Element
    if self->size = 0
    then self.error("Seq(Element).at: empty sequence.")
    else if n <= 0 
         then self->head
         else self->tail.at(n - 1)
         end
    end
  end  
\end{verbatim}
The operation {\tt butLast} returns all elements in a sequence but the last element. It could
have been defined as follows, note the use of {\tt Seq\{head | tail\}} to construct a sequence
with the given head and tail:
\begin{verbatim}
context Seq(Element)
  @Operation butLast():Seq(Element)
    if self->size = 0
    then self.error("Seq(Element)::butLast: empty sequence.")
    else if self->size = 1
         then Seq{}
         else Seq{self->head | self->tail->butLast}
         end
    end
  end
\end{verbatim}
The operation {\tt contains} returns true when a sequence contains a supplied element; {\tt drop}
takes an integer and returns a sequence that is the result of dropping the supplied number of elements;
{\tt flatten} maps a sequence of sequences to a sequence:
\begin{verbatim}
context Seq(Element)
  @Operation flatten():Seq(Element)
    if self->isEmpty
    then self
    else self->head + self->tail->flatten
    end
  end
\end{verbatim}
The operation {\tt hasPrefix} takes a sequence as an argument and returns true when the receiver has a
prefix that is equal to the argument; {\tt including} takes an element and returns a new sequence, this is
the receiver if it contains to the argument or the argument prepended to the receiver; {\tt indexOf} takes 
an element and returns the index of the argument in the receiver:
\begin{verbatim}
context SeqOfElement
  @Operation indexOf(element:Element):Integer
    if self = Seq{}
    then -1
    else 
      if self->head = element
      then 0
      else self->tail->indexOf(element) + 1
      end
    end
  end
\end{verbatim}
The operation {\tt insertAt} takes an element and an index and inserts the element at the given index; 
{\tt last} returns the last element in a sequence; {\tt hasSuffix} takes a sequence as an argument and
returns true when the receiver ends with the argument:
\begin{verbatim}
context Seq(Element)
  @Operation hasSuffix(suffix):Boolean
    self->reverse->hasPrefix(suffix->reverse)
  end
\end{verbatim}
The operation {\tt head} returns the head of a non-empty sequence; {\tt includes} returns true when the 
argument is included in the receiver; {\tt isEmpty} returns true when the receiver is empty; {\tt isProperSequence}
returns true when the final element in the sequence is a pair whose tail is {\tt Seq\{\}}; {\tt map}
applies an operation to each element of a sequence, the definition of {\tt map} shows the {\em varargs} feature
of XOCL operations where the last argument may be preceded by a {\tt .} indicating that any further
supplied arguments are bundled up into a sequence and supplied as a single value:
\begin{verbatim}
context Seq(Element)
  @Operation map(message:String . args:Seq(Element)):Element
    self->collect(x | x.send(message,args))
  end
\end{verbatim}
The operation {\tt max} finds the maximum of a sequence of integers; {\tt prepend} adds an element to 
the head of a sequence; {\tt qsort} takes a binary predicate operation and sorts the receiver into an
order that satisfies the predicate using the quicksort algorithm:
\begin{verbatim}
context Seq(Element)
  @Operation qsort(pred):Seq(Element)
    if self->isEmpty
    then self
    else 
      let e = self->head
      in let pre = self->select(x | pred(x,e));
             post = self->select(x | x <> e and not pred(x,e))
         in pre->sort(pred) + Seq{e} + post->sort(pred)
         end
      end
    end
  end
\end{verbatim}
The operation {\tt ref} can be used to lookup a namespace path represented as a sequence of strings
to the element found at the path. The operation takes a sequence of namespaces as an argument;
the namespace arguments are used as the basis for the lookup, for example:
\begin{verbatim}
Seq{"Root","EMOF","Class","attributes"}->ref(Seq{Root})
\end{verbatim}
returns the attribute named {\tt attributes}. Note that namespace {\tt Root} contains itself.

The operation {\tt reverse} reverses the receiver:
\begin{verbatim}
context Seq(Element)
 @Operation reverse():Seq(Element)
    if self->isEmpty
    then Seq{}
    else self->tail->reverse + Seq{self->head}
  end
\end{verbatim}
The operation {\tt separateWith} takes a string as an argument and returns the string that is formed
by placing the argument between each element of the receiver after the element is transformed into
a string using {\tt toString}.

The operation {\tt subst} takes three arguments: {\tt new}, {\tt old} and {\tt all}; it returns the
result of replacing element {\tt old} with {\tt new} in the receiver. If {\tt all} is true then 
all elements are replaced otherwise just the first element is replaced.

The operation {\tt subSequence} takes a staring and terminating indices and returns the appropriate
subsequence; {\tt take} takes an integer argument and returns the prefix of the receiver with that number
of elements; {\tt tail} returns the tail of a non-empty sequence.

Sequences have identity in XOCL; the head and tail of a sequence can be updated using:
\begin{verbatim}
S->tail := e
\end{verbatim}
and
\begin{verbatim}
S->tail := e
\end{verbatim}
This makes sequences very flexible and can be used for efficient storage and update of large collections
(otherwise each time a sequence was updated it would be necessary to copy the sequence).

\section{A-Lists}

An {\em a-list} is a sequence of pairs; each pair has a head that is a key and a tail that is the value 
associated with the key in the a-list. A-lists are used as simple lookup tables. They are much more 
lightweight than instances of the class {\tt Table} and have the advantage that the large number of
builtin sequence operations apply to a-lists. 

The following class shows how an a-list can be used to store the age of a collection of people:
\begin{verbatim}
context Root
  @Class People
    @Attribute table : Seq(Element) end
    @Operation newPerson(name:String,age:Integer)
      self.table := table->bind(name,age)
    end
    @Operation getAge(name:String):Integer
      table->lookup(name)
    end
    @Operation hasPerson(name:String):Boolean
      table->binds(name)
    end
    @Operation birthday(name:String)
      // Assumes name is in table:
      table->set(name,table->lookup(name) + 1)
    end
  end
\end{verbatim}

\section{Iteration}

Iteration expressions in XOCL allow collections (sets and sequences) to be manipulated in a convenient
way. Iteration expressions are a shorthand for higher-order operations that take an operation as
an argument and apply the argument operation to each element of the collection in turn. As such,
iteration expressions can be viewed as sugar for the invocation of the equivaent higher-order operations.
This section defines the XOCL iteration expressions and shows how they can be defined as higher-order
operations.

A collection can be filtered using a {\tt select} expression:
\begin{verbatim}
S->select(x | e)
\end{verbatim}
where {\tt x} is a variable and {\tt e} is a predicate expression. The result is the sub-collection
of the receiver where each element {\tt y} in the sub-collection satisfies the predicate when {\tt x} is 
bound to {\tt y}. This can be defined as follows for sequences:
\begin{verbatim}
context Seq(Element)
  @Operation select(pred:Operation):Seq(Element)
    if self->isEmpty
    then self
    else 
      if pred(self->head)
      then Seq{self->head | self->tail.select(pred)}
      else self->tail.select(pred)
      end
    end
  end
\end{verbatim}
The {\tt reject} expression is like {\tt select} except that it produces all elements that fail
the predicate; here is the definition of {\tt refject} for sets:
\begin{verbatim}
context Set(Element)
  @Operation reject(pred:Operation):Set(Element)
    if self->isEmpty
    then self
    else 
      let x = self->sel
      in if pred(x)
         then self->excluding(x)->reject(pred)
         else self->excluding(x)->reject(pred)->including(x)
         end
      end
    end
  end
\end{verbatim}
The {\tt collect} expression maps a unary operation over a collection:
\begin{verbatim}
S->collect(x | e)
\end{verbatim}
It is defined for sequences as follows:
\begin{verbatim}
context Seq(Element)
  @Operation collect(map:Operation):Seq(Element)
    if not self->isEmpty
    then 
      let x = self->sel
      in self->excluding(x)->select(map)->including(map(x))
      end
    else self
    end
  end
\end{verbatim}
\label{iteration}
The expression {\tt iterate} has the form:
\begin{verbatim}
S->iterate(x y = v | e)
\end{verbatim}
This expression steps through the elements of {\tt S} and repeatedly sets the value of {\tt y} to be
the value of {\tt e} where {\tt e} may refer to {\tt x} and {tt y}. The initial value of {\tt y} is
the value {\tt v}. The following shows an example that adds up the value of a sequence of integers:
\begin{verbatim}
Seq{1,2,3,4,5}->iterate(i sum = 0 | sum + i)
\end{verbatim}
Th definition of {\tt iterate} as a higher order operation on sequences is as follows:
\begin{verbatim}
context Seq(Element)
  @Operation iterate(y:Element,map:Operation):Element
    if self->isEmpty
    then y
    else self->tail.iterate(map(self->head,y),map)
    end
  end
\end{verbatim}

\section{Variables and Scoping}

Variables in XMF are of three types: {\em slots}, {\em dynamic} and {\em lexical}. When a message is
sent to an object, all of the slots defined by the object are bound to the slot values, For example:
\begin{verbatim}
context Root
  @Class Point
    @Attribute x : Integer end
    @Attribute y : Integer end
    @Constructor(x,y) ! end
    @Operation getX():Integer
      x
    end
    @Operation getY():Integer
      y
    end
  end
\end{verbatim}
The references to {\tt x} and {\tt y} in the accessor operations defined in {\tt Point} are variables
of type {\em slot}. The class definition given for {\tt Point} is equivalent to the shorthand
definition:
\begin{verbatim}
context Root
  @Class Point
    @Attribute x : Integer (?) end
    @Attribute y : Integer (?) end
    @Constructor(x,y) ! end
  end
\end{verbatim}
where the {\tt (?)} are attribute definition modifiers. Slot variables can always be replaced with a qualified reference using {\tt self}
as in {\tt self.x} and {\tt self.y}. Using qualified references is a matter of taste: somtimes it is
more readable to use a qualified reference. Note that it is not possible to update slots using variable
syntax: {\tt x := 0}; it is always necessary to update slots using a qualified update as follows:
{\tt self.x := 0}.

Dynamic variables are typically values
associated with names in {\em name-spaces}. Dynamic variables are established when the association between
the variable name and the value is created and typically persist for the rest of the lifetime of the XMF session.
Lexical variables are typically created when values are supplied as arguments to an operation or when local
definitions are executed. The association between the lexical variable name and the value persist for the duration 
of the operation definition or the execution of the body of the local block. In both cases, as the name suggests,
variable values can change by side-effect.

A dynamic variable added to the {\tt Root} name-space has global scope because {\tt Root} is {\em imported}
everywhere. A new dynamic variable in a name-space {\tt N} is created as follows:
\begin{verbatim}
N::v := e;
\end{verbatim}
where {\tt v} is the name of the variable and {\tt e} is an expression that produces the initial value. In
order to make the names in a name-space visible you need to import the name-space. Imports can take place
{\em locally} using the form:
\begin{verbatim}
import N in e end
\end{verbatim}
or {\em globally} in a file, by placing an {\tt import N;} at the head of the file.

Lexical variables are established when arguments are passed to an operation or using a {\tt let}
expression. In both cases the variable can be referenced in the body of the expression, but not
outside the body. In both cases the variables can be updated using {\tt v := e}. Suppose we require
an operation that takes two integers and returns a pair where the head is the smallest integer and 
the tail is the other integer:
\begin{verbatim}
context Root
  @Operation orderedPair(x,y)
    let min = 0;
        max = 0
    in if x < y then min := x else min := y end;
       if x > y then max := x else max := y end;
       Seq{min | max}
    end
  end
\end{verbatim}
The definition of {\tt orderedPair} shows how a {\tt let} expression can introduce a number of
variables (in this case {\tt min} and {\tt max}). If the {\tt let}-bindings are separated using {\tt ;}
then the bindings are established {\em in-parallel} meaning that the variables cannot affect each
other (i.e. the value for {\tt max} cannot refer to {\tt min} and vice versa). If the bindings are
separated using {\tt then} they are established {\tt in-series} meaning that values in subsequent
bindings can refer to variables in earlier bindings, for example:
\begin{verbatim}
context Root
  @Operation orderedPair(x,y)
    let min = if x < y then x else y end then
        max = if min = x then y else x end
    in Seq{min | max}
    end
  end
\end{verbatim}

\section{Looping and Find}

XMF provides {\tt While} and {\tt For} for looping through collections and provides {\tt Find} for
selecting an element in a collection that satisfies a condition. This section provides examples 
of these constructs.

A {\tt While} loop performs an action until a condition is satisfied (not a named element may use
a symbol for its name so we ensure the name is a string using the {\tt toString} operation):
\begin{verbatim}
context Root
  @Operation findElement(N:Set(NamedElement),name:String)
    let found = null
    in @While not N->isEmpty do
         let n = N->sel
         in if n.name().toString() = name
            then found := n
            else N := N->excluding(n)
            end
         end
       end;
       found
    end
  end
\end{verbatim}
It is often the case that {\tt While} loops are used to iterate through a collection. This pattern
is captured by a {\tt For} loop:
\begin{verbatim}
context Root
  @Operation findElement(N:Set(NamedElement),name:String)
    let found = null
    in @For n in N do
         if n.name().toString() = name
         then found := n
         end
       end;
       found
    end
  end
\end{verbatim} 
In general a {\tt For} loop {\tt @For x in S do e end} is equivalent to the following
{\tt While} loop:
\begin{verbatim}
let forColl = S;
    isFirst = true
in @While not forColl->isEmpty do
     let x = forColl->sel
     in forColl := forColl->excluding(x);
        let isLast = forColl->isEmpty
        in e;
           isFirst := false
        end
     end
   end
end
\end{verbatim}
where the variables {\tt forColl}, {\tt isFirst} and {\tt isLast} are scoped over the
body of the loop {\tt e}. These can be useful if we want the body action to depend on
whether this is the first or last iteration, for example turning a sequence into a 
string:
\begin{verbatim}
context Seq(Operation)
  @Operation toString()
    let s = "Seq{"
    in @For e in self do
         s := s + e.toString();
         if not isLast then s := s + "," end
       end;
       s + "}"
    end
  end
\end{verbatim}
A {\tt For} loop may return a result. The keyword {\tt do} states that the body of the
{\tt For} loop is an action and that the result of performing the entire loop will be
ignored when the loop exits. Alternatively, the keyword {\tt produce} states that the
loop will return a sequence of values. The values are the results returned by the loop
body each time it is performed. For example, suppose we want to calculate the sequence
of names from a sequence of people:
\begin{verbatim}
context Root
  @Operation getNames(people:Seq(Person)):Seq(String)
    @For person in people produce 
      person.name 
    end
  end
\end{verbatim}
The keyword {\tt in} is a {\tt For}-loop {\it directive}. After {\tt in} the loop
expects one or more collections. The {\tt in} directive supports multiple variables.
This feature is useful when stepping through multiple collections in sync, as in:
\begin{verbatim}
context Root
  @Operation createTable(names:Seq(String),addresses:Seq(String),telNos:Seq(String))
    @For name,address,telNo in names,addresses,telNos produce
      Seq{name,address,telNo}
    end
  end
\end{verbatim}
A {\tt For}-loop can be used to iterate through a table. Often we want to iterate
either through the table values or the table keys. If we use the {\tt in}
directive to iterate through either of these then we will create an intermediate
collection:
\begin{verbatim}
context Root
  @Operation addToAll(n:Integer,t:Table)
    @For k in table.keys() do
      k.put(k,t.get(k) + n)
    end
  end
\end{verbatim}
The expression {\tt table.keys()} creates an intermediate collection of all the keys
in the table. The collection is not returned and cannot be referenced independently
within the body of the loop. Therefore the collection is transient. The allocation of 
transient table collections can be avoided using the {\tt For}-loop directives 
{\tt inTableValues} and {\tt inTableKeys}:
\begin{verbatim}
context Root
  @Operation addToAll(n:Integer,t:Table)
    @For k inTableKeys table do
      k.put(k,t.get(k) + n)
    end
  end
\end{verbatim}
            
\section{Operations}

XMF operations are used to implement both procedures and functions. An operation has an optional
name, some parameters, a return type and a body. Operations are objects with internal state;
part of the internal state is the name, parameter information, type and body. Operations
also have {\it property lists} that can be used to attach information to the operation for
use by XMF programs.

Operations can be created and stored in XMF data items. In particular, operations can be added
to name spaces and then referenced via the name space (either where the name space is {\it imported}
or directly by giving the path to the operation). We have seen many examples of adding
operations to the name space called {\tt Root} in earlier parts of this primer. The syntax:
\begin{verbatim}
context Root
  @Operation add(x,y) x + y end
\end{verbatim}
can occur at the top-level of an XMF source file, compiled and loaded. It is equivalent to the
following expression:
\begin{verbatim}
Root.add(@Operation add(x,y) x + y end);
\end{verbatim}
Unlike the {\tt context} expression, the call to {\tt add} may occur anywhere in XMF code.

Operations are performed by sending them a message {\tt invoke} with two arguments: the
value of {\tt self} (or {\it target}) to be used in the body of the operation and a sequence of 
argument values. The target of the invocation is important because it provides the value 
of {\tt self} in the ody of the operation and supplies the values of the {\it slot}-bound
variables. The {\tt add} operation can be invoked by:
\begin{verbatim}
add.invoke(null,Seq{1,2})
\end{verbatim}
produces the value {\tt 3}. Note that in this case there is no reference to {\tt self}
or slot-bound variables in the body and therefore the target of the invocation is {\tt null}.
A shorthand for invocation is provided:
\begin{verbatim}
add(1,2)
\end{verbatim}
however, note that no target can be supplied with the shorthand. In this case the target will
default to the value of {\tt self} that was in scope when the operation was created.

Lexically bound variables that are scoped over an operation are available within the body
of the operation event though the operation is returned from the lexical context. This is
often referred to as {\it closing} the lexical variable into the operation (or {\it closure}).
This feature is very useful when generating behaviour that differs only in terms of
context. Suppose that transition machine states have an action that is implemented as an
operation and that the action is to be performed when the state is entered:
\begin{verbatim}
context StateMachines
  @Class State
    @Attribute name : String end
    @Attribute action : Operation end
    @Constructor(name,action) end
    @Operation enter()
      action()
    end
  end
\end{verbatim}

      

\section{Arrays and Vectors}

\section{NameSpaces, Bindings and Named Elements}

\section{Documentation}

\section{Exception Handling}

When an error occurs in XOCL, the source of the error {\em throws} an exception. The exception is a value
that, in general, contains a description of the problem and any data that might help explain the
reason why the problem occurred. 

An exception is thrown to the most recently established handler; intermediate code is discarded. If
no handler exists then the XOCL VM will terminate. In most cases, the exception is {\em caught} by a user
defined handler or, for example in the case of the XMF console, a handler established by the top
level command interpreter.

When an exception is caught, the handler can inspect the contents of the exception and decide what to do.
For example it may be necessary to re-throw the exception to the next-most recently established handler,
since it cannot be dealt with. On the other hand, it is usual to catch the exception, print a message,
patch up the problem, or just give up on the requested action.

\appendix

\section{XOCL Grammar}

\small
\begin{verbatim}
  AName ::= Name | Drop. 

  Apply ::= PathExp Args | KeyArgs .

  ArithExp ::= UpdateExp [ ArithOp ArithExp ].

  ArithOp ::= '+' | '-' | '*' | '/'.

  Args ::= '(' (')' | Exp (',' Exp)* ')'.

  AtExp ::= '@' AtPath @ 'end'.

  AtPath ::= Name ('::' Name)*.

  Atom ::= VarExp | 'self' | Str | Int | IfExp | Bool | LetExp | 
           CollExp | AtExp | Drop | Lift | '(' Exp ')' | Throw | Try | 
           ImportIn | Float.

  AtomicPattern ::= Varp | Constp | Objectp | Consp | Keywordp.
    
  Binding ::= AName '=' LogicalExp.

  Bindings ::= Binding (';' Binding)*.

  Bool ::= 'true' | 'false'.

  CollExp ::= SetExp | SeqExp. 

  CompareExp ::= ArithExp [ CompareOp CompareExp ].

  CompareOp ::= '='| '<' | '>' | '<>' | '>=' | '<='. 

  CompilationUnit ::= ParserImport* Import* (Def | TopLevelExp)* EOF.
  
  Consp ::= Pairp | Seqp | Emptyp.

  Constp ::=  Int | Str | Bool | Expp.

  Def ::= 'context' PathExp Exp. 
 
  Drop ::= '<' Exp '>'. 

  EmptyColl ::= Name '{' '}'.

  Emptyp ::= Name '{' '}'.

  Exp ::= OrderedExp.
    
  Expp ::= '[' Exp ']'.
    
  Float ::= Int '.' Int.

  Import ::= 'import' TopLevelExp.

  ImportIn ::= 'import' Exp 'in' Exp 'end'.

  ParserImport ::= 'parserImport' Name ('::' Name)* ';' ImportAt.

  IfExp ::= 'if'Exp 'then'Exp IfTail.

  IfTail ::= 'else' Exp 'end' | 'elseif' Exp 'then' Exp IfTail | 'end'.

  KeyArgs ::= '[' (']' | KeyArg (',' KeyArg)* ']'.
    
  KeyArg ::= Name '=' Exp.
    
  Keywordp ::= Name ('::' Name)* '[' Keyps ']'.
    
  Keyps ::= Keyp (',' Keyp)* | .
    
  Keyp ::= Name '=' Pattern.
    
  Lift ::= '[|' Exp '|]'.

  LetBody ::= 'in'Exp| 'then' Bindings LetBody.

  LetExp ::= 'let'Bindings LetBody 'end'.
    
  LogicalExp ::= NotExp [ LogicalOp LogicalExp ].

  LogicalOp ::= 'and' | 'or' | 'implies'.

  NonEmptySeq ::= Name '{' Exp ((',' Exp)* '}' | '|' Exp '}').
    
  NonEmptyColl ::= Name '{' Exp (',' Exp)* '}'.

  NotExp ::= CompareExp | 'not' CompareExp.

  Objectp ::= Name ('::' Name)* '(' Patterns ')'.

  OrderedExp ::= LogicalExp [ ';' OrderedExp ].

  OptionallyArgs ::= Args | .

  Pairp ::= Name '{' Pattern '|' Pattern '}'.
    
  PathExp ::= Atom [ '::' AName ('::' AName)* ].

  Pattern ::= AtomicPattern ('->' Name '(' Pattern ')')* ('when'  Exp  | ).

  Patterns ::= Pattern (',' Pattern)* | .
    
  RefExp ::= Apply 
      (
        '->' 
         (
            'iterate' '(' AName AName '=' Exp '|' Exp ')' 
          | 
           AName 
            ( 
              OptionallyArgs 
            | 
              '(' AName '|' Exp ')'  
            |
            
            ) 
          ) 
     
        | 
          '.' AName 
          (
            Args
          |   
          
          )   
      )*. 
  
    Seqp ::=  Name '{' Pattern SeqpTail.

    SeqpTail ::= ',' Pattern SeqpTail | '}'.
    
    SeqExp ::= EmptyColl | NonEmptySeq.

    SetExp ::= EmptyColl | NonEmptyColl. 

    Throw ::= 'throw' LogicalExp.

    TopLevelExp ::= LogicalExp ';'.

    Try ::= 'try' Exp 'catch' '(' Name ')' Exp 'end'.

    UpdateExp ::= RefExp (':=' ! LogicalExp | ).

    VarExp ::= Name Token.

    Varp ::= AName ('=' Pattern | ) (':' Exp | ).

\end{verbatim}


\end{document}