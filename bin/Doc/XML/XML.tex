\documentclass{article}

\newenvironment{attribute}[1]{\noindent{\tt #1}\begin{quote}}{\end{quote}}
\newenvironment{operation}[1]{\noindent{\tt #1}\begin{quote}}{\end{quote}}

\title{XML Support in XMF}

\author{Tony Clark}

\begin{document}

\maketitle

\section{Introduction}

XML is a standard data format for storage and application communication. XMF
provides extensive support for XML: models can be written and read using the 
XMF XML format; XML parsers can be constructed that read XML input and
synthesize XMF values; XMF models can be annotated with mappings describing
how to export instances of the models in XML format; XMF contains a model of
XML allowing the features of XMF to be used to construct, manipulate and
transform XML documents.

\section{XML Model}

This section defines the classes defined in the package {\tt XML}.

\subsection{\tt Document}
Represents an XML document containing single root element node. A document is
an XMF resource and therefore has a resource name string that describes the
location of the document (empty if unknown). The following attributes are defined:\\\\
\begin{attribute}{resourceName:String}
Defines the name of the resource containing the document.
\end{attribute}
\begin{attribute}{root:Node}
The root node of the document.
\end{attribute}
The following operations are defined:\\\\
\begin{operation}{Document(resourceName:String)}
Constructs an instance.
\end{operation}
\begin{operation}{Document(resourceName:String,root:Node)}
Constructs an instance.
\end{operation}
\begin{operation}{print()}
Prints the XML document to its resource. An error occurs if no resource is set.
\end{operation}
\begin{operation}{print(out:OutputChannel)}
Prints the XML document to the supplied output channel.
\end{operation}
\begin{operation}{pprint(out:OutputChannel)}
Prints the XML document to the supplied output channel using indentation to 
show nesting.
\end{operation}
\begin{operation}{stripWhiteSpace():Document}
Removes any occurrences of {\tt Text} elements in the document that contain
just white-pace characters. Returns a fresh copy of the document containing
no white-space.
\end{operation}
\begin{operation}{reduce():Element}
Translates an XML document that conforms to the XMF.dtd (see appendix \ref{DTD}) to
the appropriate XMF element.
\end{operation}

\subsection{\tt Node}
An abstract class that is used as the super class of all document nodes.

\subsection{\tt Element}
A sub-class of {\tt Node} that has a tag, a set of attributes and a sequence
of child nodes. The following attributes are defined:\\\\
\begin{attribute}{tag:String}
The tag of the element.
\end{attribute}
\begin{attribute}{attributes:Set(Attribute)}
The attributes for the element.
\end{attribute}
\begin{attribute}{children:Seq(Node)}
The child nodes of the element.
\end{attribute}
The following operations are defined for elements:\\\\
\begin{operation}{filter(tag:String):Seq(Element)}
Returns the sequence of children with the given tag in the order that
they occur in the element.
\end{operation}
\begin{operation}{getAtt(name:String):String}
Returns the value of the attribute with the given name. Throws an exception if
no attribute with the given name exists.
\end{operation}
\begin{operation}{hasAtt(name:String):Boolean}
Tests whether the element has an attribute with the given name.
\end{operation}
\begin{operation}{print(out:OutputChannel)}
Prints the element to the supplied output channel.
\end{operation}
\begin{operation}{put(name:String,value:String)}
Sets the value of the attribute with the given name. Throws an exception if
no attribute with the given name exists.
\end{operation}
\subsection{\tt Attribute}
An XML attribute that consists of a name and a value. Both the name
and value are strings.

\subsection{\tt Text}
A sub-class of {\tt Node} that represents text. An instance of this class has
a single attribute {\tt text} of type string.


\section{XML Input and Output}

\section{Parsing XML Files}

\section{XML Mappings}

\appendix

\section{The XMF DTD}

\label{DTD}

\begin{verbatim}
<!DEFINE Value 
  (Boolean     | 
   Integer     | 
   String      | 
   Object      | 
   IdRef       | 
   Set         |
   Seq         | 
   Null        | 
   Operation   |
   EmptySeq)>

<!ELEMENT Boolean ()>
<!ATTLIST Boolean value PCDATA #REQUIRED> 

<!ELEMENT EmptySeq ()>

<!ELEMENT IdRef ()>
<!ATTLIST IdRef id ID #REQUIRED> 

<!ELEMENT Integer ()>
<!ATTLIST Integer value PCDATA #REQUIRED> 

<!ELEMENT NameSpaceRef ()>
<!ATTLIST NameSpaceRef name PCDATA #REQUIRED> 

<!ELEMENT Null ()>

<!ELEMENT Object (Ref Slot*)>
<!ATTLIST Object id ID #REQUIRED> 

<!ELEMENT Operation ()>
<!ATTLIST Operation name PCDATA #REQUIRED> 

<!ELEMENT Ref (NameSpaceRef*)>
<!ATTLIST Ref root PCDATA #REQUIRED> 

<!ELEMENT Set (Value)*>

<!ELEMENT Seq (Value Value)>

<!ELEMENT Slot (Value)>
<!ATTLIST Slot name PCDATA #REQUIRED> 

<!ELEMENT String ()>
<!ATTLIST String value PCDATA #REQUIRED> 
\end{verbatim}

\end{document}